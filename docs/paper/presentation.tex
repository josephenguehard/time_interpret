\section{Presentation of the library}
\label{sec:presentation}

We provide in this section an introduction to the \texttt{\detokenize{time_interpret}} library.
Please also refer to the documentation.

\texttt{\detokenize{time_interpret}} is primarily composed of 4 different parts: attribution methods, datasets,
evaluation tools (metrics) and deep learning models.
We present below a short description of the components in each of these parts.


\paragraph{Attribution methods.}

Attribution methods constitute the core of \texttt{\detokenize{time_interpret}}.
In this part of the library, we regrouped many methods which have been recently published.
Similarly to Captum~\citep{kokhlikyan2020captum}, each method can be called like this:

\begin{lstlisting}[language=Python, caption=Attribution loading example, label={lst:attr}]
from tint.attr import TemporalIntegratedGradients

explainer = TemporalIntegratedGradients(
    model
)
attr = explainer.attribute(inputs)
\end{lstlisting}

where $``$model$"$ is a PyTorch model, and $``$inputs$"$ is an inputs' tensor.

We provide in this library several methods:

\begin{itemize}
    \item \textbf{AugmentedOcclusion}.
        This method improves upon the original Occlusion method from captum~\url{https://captum.ai/api/occlusion.html}
        by allowing to sample the baseline from a bootstrapped distribution.
        By selecting a distribution close to the inputs, the resulting occulted data should be close to actual data,
        limiting the amount of out of distribution samples.
        This method was originally proposed by~\citep{tonekaboni2020went}, Section 4.
    \item \textbf{BayesLime, BayesKernelShap}.
        These two methods, originally proposed by~\citep{slack2021reliable}, extend respectively
        LIME~\citep{ribeiro2016should} and KernelSHAP~\citep{lundberg2017unified}, by replacing the underlying
        linear regression model with a bayesian linear regression, allowing the method to model uncertainty in
        explainability by outputting credible intervals in addition to the feature attributions.
    \item \textbf{DiscretetizedIntegratedGradients (DIG)}.
        DIG~\citep{sanyal2021discretized} was designed to interpret predictions made by language models.
        It builds upon the original Integrated Gradients method by generating discretized paths, hopping from one
        word to another, instead of using straight lines.
        This way, it aims to create a path which takes into account the discreteness of the embedding space.
    \item \textbf{DynaMask}.
        This method, introduced by~\citep{crabbe2021explaining}, is an adaptation of a perturbation-based method
        developed in~\citep{fong2017interpretable, fong2019understanding}, to handle time-series data.
        As such, it consists in perturbing a temporal data by replacing some of it with an average in time.
        The mask used to choose which data should be preserved and which should be replaced is learnt in order to either
        preserve the original prediction with a minimum amount of unmasked data, or change the original prediction with
        a small amount of masked data.
        Either way, the learnt mask can then be used to discriminate between important features and others.
    \item \textbf{ExtremalMask}.
        This method~\citep{enguehard2023learning} consists in a generalisation of DynaMask, which learns not only the
        mask, but also the associated perturbation, instead of replacing perturbed data with a predetermined average in
        time.
        Learning perturbations allows this method to take into account eventual long term dependencies, such as
        temporal regularities.
    \item \textbf{Fit}.
        Originally proposed by~\citep{tonekaboni2020went}, this method aims to understand which feature is important by
        quantifying the shift in the predictive distribution over time.
        An important feature is then one which contributes significantly to the distributional shift.
    \item \textbf{LofLime, LofKernelShap}.
        Novel method.
        Please see Section~\ref{sec:methods} for more details.
    \item \textbf{NonLinearitiesTunnel}.
        Novel method.
        Please see Section~\ref{sec:methods} for more details.
    \item \textbf{Retain}.
        This method~\citep{choi2016retain} uses two RNNs whose outputs are then used as keys and queries in an attention
        mechanism, also using the original embeddings of the input as values.
        This attention mechanism can then be used to explain which feature was important to make a specific prediction.
    \item \textbf{SequentialIntegratedGradients (SIG)}.
        SIG~\citep{enguehard2023sequential} is an adaptation of the Integrated Gradients method to sequential data.
        It modifies the baseline by only masking one element of a sequence at a time, and computing the corresponding
        feature attribution.
        By doing so, it allows the baseline to be closer to the original sequence.
    \item \textbf{TemporalOcclusion}.
        Originally proposed by~\citep{tonekaboni2020went}, this method modifies the Occlusion method from
        Captum~\url{https://captum.ai/api/occlusion.html} by only masking the last input data in time, preserving the
        previous inputs.
    \item \textbf{TemporalAugmentedOcclusion}.
        This method combines TemporalOcclusion and AugmentedOcclusion.
        As such, it only masks the last input in time, replacing it with samples from a bootstrapped distribution.
    \item \textbf{TemporalIntegratedGradients}.
        Novel method.
        Please see Section~\ref{sec:methods} for more details.
    \item \textbf{TimeForwardTunnel}.
        Novel method.
        Please see Section~\ref{sec:methods} for more details.

\end{itemize}


\paragraph{Datasets}

As part of \texttt{\detokenize{time_interpret}}, we include a collection of datasets which can be readily used:

\begin{lstlisting}[language=Python, caption=Dataset loading example, label={lst:datasets}]
from tint.datasets import Arma

arma = Arma()
arma.download()  # This method generates the dataset

inputs = arma.preprocess()["x"]
true_saliency = arma.true_saliency(dim=1)
\end{lstlisting}

All these datasets are implemented using the DataModule from PyTorch Lightning.
We provide the following datasets:

\begin{itemize}
    \item \textbf{Arma}.
        This dataset was introduced by~\citep{crabbe2021explaining}.
        It relies on an ARMA process to generate time features $x_{ti}$, which are inputs combined by white-box
        regressor f.
        This regressor uses only part of the data, so we can know in advance which feature is salient, in order to
        evaluate a features attributions method.
        As a result, this dataset can be used to provide an initial comparison of various methods on time-series data.
    \item \textbf{BioBank}.
        BioBank~\citep{sudlow2015uk} is a large dataset based on the UK population.
        As such, it can be used to predict a specific condition based on a number of patients' data.
        In the default setting, this dataset can be used to train a model to predict the risk of developing type-II
        diabetes, but other conditions could be predicted.
        We also provide a cutoff before the onset of the condition, only using features before that time, as well as
        a script to train FastText~\citep{bojanowski2017enriching} embeddings on the medical codes.
        The dataset can also be discretized, grouping medical codes into time intervals such as years.
        To access the data, a formal application must be made, please see
        \url{https://www.ukbiobank.ac.uk/enable-your-research}.
    \item \textbf{Hawkes}.
        Hawkes processes are a specific type of temporal point processes (TPP).
        The probability of an event of type k happening at time t is conditioned by a specific intensity function:
        \[ \lambda^*_k(t) = \mu_k + \sum_{n=1}^K \alpha_{kn} \sum_{t_i^n < t} \exp \left[ -\beta_{kn} (t - t_i^n) \right] \]
        with $\bm{\alpha}$, $\bm{\beta}$ and $\bm{\mu}$ being parameters of the process.
        As a result, an event has greater chance of happening if an event already happened previously, depending on the
        values of $\bm{\alpha}$ and $\bm{\beta}$.
        If no event previously happened, the base probability of an event happening is determined by $\bm{\mu}$.
        In this case, this intensity can therefore be used as the true saliency of the features.
        An important consequence for this dataset is that the true saliency is \textbf{dependent in time}.
        Indeed, an event can be very important for another one close in time, but also irrelevant for other events.
        As a result, the true saliency has two temporal dimensions: for each time, it gives the importance of each
        temporal event.
    \item \textbf{Hidden Markov model (HMM)}.
        This dataset was introduced by~\citep{crabbe2021explaining}.
        It consists in a 2-state hidden Markov model, with a features vector of size 3.
        For each time, only one of the three features conditions the label: either the second or the third, depending
        on the state.
        The first feature is always not salient.
        For this dataset, the true salient features are therefore also known.
    \item \textbf{Mimic-III}.
        Mimic-III~\citep{johnson2016mimic} consists of ICU data points.
        This dataset provides two tasks: predicting in-hospital mortality (static label) and predicting the
        average blood pressure (temporal label).
        The processing of this dataset was done following~\citep{tonekaboni2020went, crabbe2021explaining}.
\end{itemize}

\newpage

\paragraph{Metrics}

Metrics are an important component of \texttt{\detokenize{time_interpret}}, as they provide elements of comparison
between feature attribution methods in two scenarios: when the true salient features are known, and when they are
unknown.

Both metrics can be imported this way:

\begin{lstlisting}[language=Python, caption=Attribution evaluation example, label={lst:metrics}]
from tint.metrics import accuracy
from tint.metrics.white_box import aup

print(f"{aup(attr, true_saliency):.4}")
print(f"{accuracy(model, input, attr):.4}")
\end{lstlisting}

As we can see on Snippet~\ref{lst:metrics}, these metrics have a different behavior in each scenario.
In the case where the true salient features are known, the metrics are straightforward: they directly compare some
attributions with the truth.
\texttt{\detokenize{time_interpret}} provides the following metrics: area under precision (AUP), area under recall (AUR),
area under precision recall curve (AUPRC), Roc-Auc, mean average error (MAE), mean square error (MSE), root mean square
error (RMSE).
Following~\citep{crabbe2021explaining}, we also provide the Entropy and Information metrics.

In the case where the true salient features are not known, we draw from the work
of~\citep{shrikumar2017learning,deyoung2019eraser,crabbe2021explaining} and propose a list of metrics following a
similar pattern.
Each metric is computed by masking the top or bottom x\% most important features.
We then compute predictions by the model to be explained on this masked data.
The shift in the resulting predictions provides information on how important the masked data actually was.

Moreover, in order to avoid creating out of distribution data by masking some of the features, we allow the user to
pass a custom baseline to the metric.
It is also possible to draw baselines from a distribution, to add gaussian noise to the masked inputs, and to compute
predictions multiple times to produce consistent results.
In the last case, a batch size can be passed to compute these predictions in parallel and reduce time complexity.

Furthermore, it is also possible to pass a weight function to the metric: the shift in prediction given some masked
data influences the metric proportionally to some weight.
We provide two weights functions: $``$lime-weight$"$ weights the results according to a cosine or euclidean distance
between the masked data and the original one.
$``$lof-weight$"$ weights the results according to a Local Outlier Factor score, to reduce the influence of potential
outliers.

We provide the following metrics:

\begin{itemize}
    \item \textbf{accuracy}.
        This metric measures by how much the accuracy of a model drops when
        removing or only keeping the topk most important features.
        Lower is better.
        A custom threshold can also be passed to discriminate between positive and negative predictions.
    \item \textbf{comprehensiveness}.
        This measures by how much the predicted class probability changes when removing the topk most important
        features.
        Higher is better.
    \item \textbf{cross-entropy}.
        This metric measures the cross-entropy between the outputs of the model using the original inputs and perturbed
        inputs by removing or only keeping the topk most important features.
        Higher is better.
    \item \textbf{Log-odds}.
        This log-odds measures the average difference of the negative logarithmic probabilities on the predicted class
        when removing the topk most important features.
        Lower is better.
    \item \textbf{mae}.
        This metric measures the mean absolute error between the outputs of the model using the original inputs and
        perturbed inputs by removing or only keeping the topk most important features.
        Lower is better.
    \item \textbf{mse}.
        This metric measures the mean square error between the outputs of the model using the original inputs and
        perturbed inputs by removing or only keeping the topk most important features.
        Lower is better.
    \item \textbf{sufficiency}.
        This sufficiency measures by how much the predicted class probability changes when keeping only the topk most
        important features.
        Lower is better.
\end{itemize}

Finally, we provide an additional metric, $``$lipschitz-max$"$, measuring the stability of a feature attribution method
on small variations of the input data.
However, sensitivity-max should be preferred, as a more robust method.
Please see \url{https://captum.ai/api/metrics.html#sensitivity} for more details.
We still provide it as it has been used in some research and could be useful to reproduce experiments.


\paragraph{Models}

We provide several PyTorch models: MLP, CNN, RNN and TransformerEncoder, as well as some language models
from HuggingFace: Bert, DistilBert and RoBerta.
These models can be easily wrapped into a PyTorch-lightning model as such:

\begin{lstlisting}[language=Python, caption=Model definition example, label={lst:model}]
from tint.models import MLP, Net

mlp = MLP(units=[5, 10, 1])
net = Net(
    mlp, loss="cross_entropy", optim="adam"
)
\end{lstlisting}
